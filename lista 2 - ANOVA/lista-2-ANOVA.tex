% Options for packages loaded elsewhere
\PassOptionsToPackage{unicode}{hyperref}
\PassOptionsToPackage{hyphens}{url}
%
\documentclass[
]{article}
\usepackage{amsmath,amssymb}
\usepackage{lmodern}
\usepackage{iftex}
\ifPDFTeX
  \usepackage[T1]{fontenc}
  \usepackage[utf8]{inputenc}
  \usepackage{textcomp} % provide euro and other symbols
\else % if luatex or xetex
  \usepackage{unicode-math}
  \defaultfontfeatures{Scale=MatchLowercase}
  \defaultfontfeatures[\rmfamily]{Ligatures=TeX,Scale=1}
\fi
% Use upquote if available, for straight quotes in verbatim environments
\IfFileExists{upquote.sty}{\usepackage{upquote}}{}
\IfFileExists{microtype.sty}{% use microtype if available
  \usepackage[]{microtype}
  \UseMicrotypeSet[protrusion]{basicmath} % disable protrusion for tt fonts
}{}
\makeatletter
\@ifundefined{KOMAClassName}{% if non-KOMA class
  \IfFileExists{parskip.sty}{%
    \usepackage{parskip}
  }{% else
    \setlength{\parindent}{0pt}
    \setlength{\parskip}{6pt plus 2pt minus 1pt}}
}{% if KOMA class
  \KOMAoptions{parskip=half}}
\makeatother
\usepackage{xcolor}
\IfFileExists{xurl.sty}{\usepackage{xurl}}{} % add URL line breaks if available
\IfFileExists{bookmark.sty}{\usepackage{bookmark}}{\usepackage{hyperref}}
\hypersetup{
  pdftitle={Exercício de laboratorio 2},
  pdfauthor={César A. Galvão - 19/0011572},
  hidelinks,
  pdfcreator={LaTeX via pandoc}}
\urlstyle{same} % disable monospaced font for URLs
\usepackage[margin=1in]{geometry}
\usepackage{color}
\usepackage{fancyvrb}
\newcommand{\VerbBar}{|}
\newcommand{\VERB}{\Verb[commandchars=\\\{\}]}
\DefineVerbatimEnvironment{Highlighting}{Verbatim}{commandchars=\\\{\}}
% Add ',fontsize=\small' for more characters per line
\usepackage{framed}
\definecolor{shadecolor}{RGB}{248,248,248}
\newenvironment{Shaded}{\begin{snugshade}}{\end{snugshade}}
\newcommand{\AlertTok}[1]{\textcolor[rgb]{0.94,0.16,0.16}{#1}}
\newcommand{\AnnotationTok}[1]{\textcolor[rgb]{0.56,0.35,0.01}{\textbf{\textit{#1}}}}
\newcommand{\AttributeTok}[1]{\textcolor[rgb]{0.77,0.63,0.00}{#1}}
\newcommand{\BaseNTok}[1]{\textcolor[rgb]{0.00,0.00,0.81}{#1}}
\newcommand{\BuiltInTok}[1]{#1}
\newcommand{\CharTok}[1]{\textcolor[rgb]{0.31,0.60,0.02}{#1}}
\newcommand{\CommentTok}[1]{\textcolor[rgb]{0.56,0.35,0.01}{\textit{#1}}}
\newcommand{\CommentVarTok}[1]{\textcolor[rgb]{0.56,0.35,0.01}{\textbf{\textit{#1}}}}
\newcommand{\ConstantTok}[1]{\textcolor[rgb]{0.00,0.00,0.00}{#1}}
\newcommand{\ControlFlowTok}[1]{\textcolor[rgb]{0.13,0.29,0.53}{\textbf{#1}}}
\newcommand{\DataTypeTok}[1]{\textcolor[rgb]{0.13,0.29,0.53}{#1}}
\newcommand{\DecValTok}[1]{\textcolor[rgb]{0.00,0.00,0.81}{#1}}
\newcommand{\DocumentationTok}[1]{\textcolor[rgb]{0.56,0.35,0.01}{\textbf{\textit{#1}}}}
\newcommand{\ErrorTok}[1]{\textcolor[rgb]{0.64,0.00,0.00}{\textbf{#1}}}
\newcommand{\ExtensionTok}[1]{#1}
\newcommand{\FloatTok}[1]{\textcolor[rgb]{0.00,0.00,0.81}{#1}}
\newcommand{\FunctionTok}[1]{\textcolor[rgb]{0.00,0.00,0.00}{#1}}
\newcommand{\ImportTok}[1]{#1}
\newcommand{\InformationTok}[1]{\textcolor[rgb]{0.56,0.35,0.01}{\textbf{\textit{#1}}}}
\newcommand{\KeywordTok}[1]{\textcolor[rgb]{0.13,0.29,0.53}{\textbf{#1}}}
\newcommand{\NormalTok}[1]{#1}
\newcommand{\OperatorTok}[1]{\textcolor[rgb]{0.81,0.36,0.00}{\textbf{#1}}}
\newcommand{\OtherTok}[1]{\textcolor[rgb]{0.56,0.35,0.01}{#1}}
\newcommand{\PreprocessorTok}[1]{\textcolor[rgb]{0.56,0.35,0.01}{\textit{#1}}}
\newcommand{\RegionMarkerTok}[1]{#1}
\newcommand{\SpecialCharTok}[1]{\textcolor[rgb]{0.00,0.00,0.00}{#1}}
\newcommand{\SpecialStringTok}[1]{\textcolor[rgb]{0.31,0.60,0.02}{#1}}
\newcommand{\StringTok}[1]{\textcolor[rgb]{0.31,0.60,0.02}{#1}}
\newcommand{\VariableTok}[1]{\textcolor[rgb]{0.00,0.00,0.00}{#1}}
\newcommand{\VerbatimStringTok}[1]{\textcolor[rgb]{0.31,0.60,0.02}{#1}}
\newcommand{\WarningTok}[1]{\textcolor[rgb]{0.56,0.35,0.01}{\textbf{\textit{#1}}}}
\usepackage{graphicx}
\makeatletter
\def\maxwidth{\ifdim\Gin@nat@width>\linewidth\linewidth\else\Gin@nat@width\fi}
\def\maxheight{\ifdim\Gin@nat@height>\textheight\textheight\else\Gin@nat@height\fi}
\makeatother
% Scale images if necessary, so that they will not overflow the page
% margins by default, and it is still possible to overwrite the defaults
% using explicit options in \includegraphics[width, height, ...]{}
\setkeys{Gin}{width=\maxwidth,height=\maxheight,keepaspectratio}
% Set default figure placement to htbp
\makeatletter
\def\fps@figure{htbp}
\makeatother
\setlength{\emergencystretch}{3em} % prevent overfull lines
\providecommand{\tightlist}{%
  \setlength{\itemsep}{0pt}\setlength{\parskip}{0pt}}
\setcounter{secnumdepth}{5}
\usepackage{helvet} \renewcommand\familydefault{\sfdefault}
\usepackage{booktabs}
\usepackage{longtable}
\usepackage{array}
\usepackage{multirow}
\usepackage{wrapfig}
\usepackage{float}
\usepackage{colortbl}
\usepackage{pdflscape}
\usepackage{tabu}
\usepackage{threeparttable}
\usepackage{threeparttablex}
\usepackage[normalem]{ulem}
\usepackage{makecell}
\usepackage{xcolor}
\ifLuaTeX
  \usepackage{selnolig}  % disable illegal ligatures
\fi

\title{Exercício de laboratorio 2}
\author{César A. Galvão - 19/0011572}
\date{2022-06-26}

\begin{document}
\maketitle

\newpage{}

{
\setcounter{tocdepth}{2}
\tableofcontents
}
\let\oldsection\section
\renewcommand\section{\clearpage\oldsection}

\hypertarget{questao-1}{%
\section{Questao 1}\label{questao-1}}

\begin{longtable}{lc}
\toprule
tipo & tempo\\
\midrule
\endfirsthead
\multicolumn{2}{@{}l}{\textit{(continued)}}\\
\toprule
tipo & tempo\\
\midrule
\endhead

\endfoot
\bottomrule
\endlastfoot
\cellcolor{gray!15}{I} & \cellcolor{gray!15}{19}\\
I & 22\\
\cellcolor{gray!15}{I} & \cellcolor{gray!15}{20}\\
I & 18\\
\cellcolor{gray!15}{I} & \cellcolor{gray!15}{25}\\
II & 20\\
\cellcolor{gray!15}{II} & \cellcolor{gray!15}{21}\\
II & 33\\
\cellcolor{gray!15}{II} & \cellcolor{gray!15}{27}\\
II & 40\\
\cellcolor{gray!15}{III} & \cellcolor{gray!15}{16}\\
III & 15\\
\cellcolor{gray!15}{III} & \cellcolor{gray!15}{18}\\
III & 26\\
\cellcolor{gray!15}{III} & \cellcolor{gray!15}{17}\\*
\end{longtable}

\hypertarget{determine-a-forma-do-modelo-e-as-hipuxf3teses-consideradas}{%
\subsection{Determine a forma do modelo e as hipóteses
consideradas}\label{determine-a-forma-do-modelo-e-as-hipuxf3teses-consideradas}}

A comparação das médias dos grupos, neste caso os tipos de circuito,
será realizada mediante análise de variância. O modelo escolhido para
tal é o modelo de efeitos, expresso na equação a seguir

\begin{equation}
  y_{ij} = \mu + \tau_i + e_{ij}, \quad i = 1, 2,..., a; \quad j = 1, 2,..., n 
\end{equation}

em que \(\mu\) é a média geral, \(\tau_i\) é a média ou efeito dos
grupos e \(e_{ij}\) é o desvio do elemento. Os grupos são indexados por
\(i\) e os indivíduos de cada grupo indexados por \(j\).

As hipóteses do teste são as seguintes: \begin{align}
  \begin{cases}
    H_0: \tau_1 = ... = \tau_a = 0, \quad \text{(O efeito de tratamento é nulo)}\\
    H_1: \exists \tau_i \neq 0
  \end{cases}
\end{align}

que equivale dizer

\begin{align}
  \begin{cases}
    H_0: \mu_1 = ... = \mu_a\\
    H_1: \exists \mu_i \neq \mu_j, \, i \neq j.
  \end{cases}
\end{align}

\hypertarget{qual-a-forma-da-estatuxedstica-de-teste-e-sua-distribuiuxe7uxe3o-amostral}{%
\subsection{Qual a forma da estatística de teste e sua distribuição
amostral?}\label{qual-a-forma-da-estatuxedstica-de-teste-e-sua-distribuiuxe7uxe3o-amostral}}

A estatística de teste é calculada mediante a média ponderada entre a
soma dos quadrados dos tratamentos e a soma dos quadrados dos resíduos
(quadrados médios dos tratamentos e dos resíduos respectivamente). Sob
\(H_0\) a estatística de teste tem distribuição \(F(a-1, an-a)\). Os
graus de liberdade correspondem aos denominadores dos quadrados médios.
Especificamente,

\begin{align}
  \frac{\frac{\text{SQTRAT}}{a-1}}{\frac{\text{SQRES}}{an-a}} = \frac{\text{QMTRAT}}{\text{QMRES}} \sim F(a-1, an-a)
\end{align}

\hypertarget{construa-a-tabela-de-anuxe1lise-de-variuxe2ncia-e-conclua-o-teste-considerando-alfa-005}{%
\subsection{Construa a tabela de análise de variância e conclua o teste
considerando alfa =
0,05}\label{construa-a-tabela-de-anuxe1lise-de-variuxe2ncia-e-conclua-o-teste-considerando-alfa-005}}

O objeto
\texttt{tabela\ \textless{}-\ aov(tempo\ \textasciitilde{}\ tipo,\ dados)}
é gerado para criação da tabela a seguir.

\begin{longtable}{cccccc}
\toprule
term & df & sumsq & meansq & statistic & p.value\\
\midrule
\endfirsthead
\multicolumn{6}{@{}l}{\textit{(continued)}}\\
\toprule
term & df & sumsq & meansq & statistic & p.value\\
\midrule
\endhead

\endfoot
\bottomrule
\endlastfoot
\cellcolor{gray!15}{tipo} & \cellcolor{gray!15}{2} & \cellcolor{gray!15}{260.9333} & \cellcolor{gray!15}{130.46667} & \cellcolor{gray!15}{4.006141} & \cellcolor{gray!15}{0.0464845}\\
Residuals & 12 & 390.8000 & 32.56667 & NA & NA\\*
\end{longtable}

Com base apenas na ANOVA, cujo p-valor é \(< 0,05\), há evidências para
rejeitar \(H_0\), ou seja, existe pelo menos uma média de grupo
diferente das demais.

\hypertarget{quais-suxe3o-as-suposiuxe7uxf5es-adotadas-para-a-anova-essas-suposiuxe7uxf5es-foram-satisfeitas-para-esse-experimento}{%
\subsection{Quais são as suposições adotadas para a ANOVA? Essas
suposições foram satisfeitas para esse
experimento?}\label{quais-suxe3o-as-suposiuxe7uxf5es-adotadas-para-a-anova-essas-suposiuxe7uxf5es-foram-satisfeitas-para-esse-experimento}}

Para o teste de análise de variâncias, considerando o modelo de efeitos,
supõe-se sobre os resíduos, elemento aleatório do lado direito da
expressão do modelo:

\begin{itemize}
\tightlist
\item
  independência;
\item
  normalidade;
\item
  homogeneidade de variâncias (homocedasticidade).
\end{itemize}

Por hipótese, supõe-se que as amostras são independentes. Não há, a
priori, como testar independência pois entende-se que isso é derivado do
desenho do experimento.

A normalidade da distribuição dos resíduos pode ser testada mediante o
teste de Shapiro-Wilk, realizada utilizando
\texttt{shapiro.test(tabela\$residuals)}, em que \texttt{tabela} é o
modelo de análise de variâncias gerado anteriormente.

\begin{longtable}{ccc}
\toprule
statistic & p.value & method\\
\midrule
\endfirsthead
\multicolumn{3}{@{}l}{\textit{(continued)}}\\
\toprule
statistic & p.value & method\\
\midrule
\endhead

\endfoot
\bottomrule
\endlastfoot
\cellcolor{gray!15}{0.9423982} & \cellcolor{gray!15}{0.4134853} & \cellcolor{gray!15}{Shapiro-Wilk normality test}\\*
\end{longtable}

O teste assume como hipótese nula a normalidade dos dados amostrais. Com
base no p-valor obtido, não há evidências para a rejeição de \(H_0\).
Isto é, supõe-se normalidade dos dados.

Quando à homocedasticidade, utiliza-se o teste de Levene. A hipótese
nula supõe homogeneidade de variâncias entre as amostras.

\begin{longtable}{ccccc}
\toprule
teste & F statistic & p.value & df & df.residual\\
\midrule
\endfirsthead
\multicolumn{5}{@{}l}{\textit{(continued)}}\\
\toprule
teste & F statistic & p.value & df & df.residual\\
\midrule
\endhead

\endfoot
\bottomrule
\endlastfoot
\cellcolor{gray!15}{Teste Levene de Homogeneidade} & \cellcolor{gray!15}{2.24147} & \cellcolor{gray!15}{0.1488948} & \cellcolor{gray!15}{2} & \cellcolor{gray!15}{12}\\*
\end{longtable}

De fato, obtém-se p-valor superior a 0.05, sugerindo a não rejeição de
\(H_0\).

\hypertarget{fauxe7a-comparauxe7uxf5es-entre-os-pares-de-muxe9dias-pelo-teste-de-tukey-e-apresente-os-resultados.}{%
\subsection{Faça comparações entre os pares de médias pelo teste de
Tukey e apresente os
resultados.}\label{fauxe7a-comparauxe7uxf5es-entre-os-pares-de-muxe9dias-pelo-teste-de-tukey-e-apresente-os-resultados.}}

Opta-se pelo teste de Tukey para comparações múltiplas de médias.
Trata-se de um teste unilateral para comparação de médias entre grupos
de tratamento. Sob \(H_0\), ou seja, a igualdade entre as médias
comparadas, a estatística de teste segue uma distribuição Tukey, cujos
parâmetros são os graus de liberdade do resíduo e o número de
comparações:

\begin{align}
  \frac{|\bar{y}_i. - \bar{y}_j.|}{\sqrt{\frac{\text{QMRES}}{n}}} \stackrel{H_0}{\sim} \text{Tukey} \left( gl. res., n^o comp. \right)
\end{align}

\begin{longtable}{cccccc}
\toprule
term & contrast & estimate & conf.low & conf.high & adj.p.value\\
\midrule
\endfirsthead
\multicolumn{6}{@{}l}{\textit{(continued)}}\\
\toprule
term & contrast & estimate & conf.low & conf.high & adj.p.value\\
\midrule
\endhead

\endfoot
\bottomrule
\endlastfoot
\cellcolor{gray!15}{tipo} & \cellcolor{gray!15}{II-I} & \cellcolor{gray!15}{7.4} & \cellcolor{gray!15}{-2.22898} & \cellcolor{gray!15}{17.0289799} & \cellcolor{gray!15}{0.1425885}\\
tipo & III-I & -2.4 & -12.02898 & 7.2289799 & 0.7876393\\
\cellcolor{gray!15}{tipo} & \cellcolor{gray!15}{III-II} & \cellcolor{gray!15}{-9.8} & \cellcolor{gray!15}{-19.42898} & \cellcolor{gray!15}{-0.1710201} & \cellcolor{gray!15}{0.0459970}\\*
\end{longtable}

Pelo teste de Tukey, há indícios para rejeição de \(H_0\) apenas quando
comparados os grupos II e III, corroborando o resultado da análise de
variâncias.

\hypertarget{construa-um-intervalo-de-confianuxe7a-para-muxe9dia-do-circuito-com-menores-tempos-considerando-gama-098}{%
\subsection{Construa um intervalo de confiança para média do circuito
com menores tempos considerando gama =
0,98}\label{construa-um-intervalo-de-confianuxe7a-para-muxe9dia-do-circuito-com-menores-tempos-considerando-gama-098}}

\begin{longtable}{cc}
\toprule
tipo & media\\
\midrule
\endfirsthead
\multicolumn{2}{@{}l}{\textit{(continued)}}\\
\toprule
tipo & media\\
\midrule
\endhead

\endfoot
\bottomrule
\endlastfoot
\cellcolor{gray!15}{I} & \cellcolor{gray!15}{20.8}\\
II & 28.2\\
\cellcolor{gray!15}{III} & \cellcolor{gray!15}{18.4}\\*
\end{longtable}

O grupo de menor média de tempo é o grupo III, cuja média é de 18,4.
Considerando que a comparação da média do grupo à média global é uma
análise de resíduos, utiliza-se como variância QMRES, pois
\(E\left(\text{QMRES}\right) = \sigma^2\). Dessa forma, calcula-se o
intervalo de confiança considerando \(\gamma = 0,98\):

\begin{align}
  IC\left( \bar{y}_i.; \gamma \right) &= \bar{y}_i. \pm t_{(an-a; 1-\alpha/2)} \cdot \sqrt{\frac{\text{QMRES}}{n}}\\
  &= \bar{y}_3. \pm t_{(15-3; 1-0,01)} \cdot \sqrt{\frac{32,56667}{5}}\\
  &= \bar{y}_3. \pm t_{(12; 0,99)} \cdot \sqrt{\frac{32,56667}{5}}\\
 IC\left( \bar{y}_3.; 0,98 \right) &= 18,4 \pm 2,68 \cdot \sqrt{\frac{32,56667}{5}}\\
  &= 18,4 \pm 6,84\\
  &= \left[ 11,55 ; 25,24 \right]
\end{align}

\hypertarget{exercuxedcio-de-simulauxe7uxe3o}{%
\section{Exercício de simulação}\label{exercuxedcio-de-simulauxe7uxe3o}}

\emph{Faça um experimento de simulação considerando \(a = 4\)
tratamentos com \(n = 4\) repetições e um valor de \(\sigma^2 = 25\).
Faça \(k = 1000\) iterações em que a hipótese nula da ANOVA seja
verdadeira e verifique a proporção de casos com pelo menos um erro do
tipo I para os testes de comparações múltiplas de médias usando as
técnicas de Tukey e Fisher e verificando se existem diferenças entre as
técnicas.}

\emph{Caso os testes de comparação múltipla sejam feitos apenas após o
teste da anova ser significativo os resultados do item anteiror são
alterados?}

São realizadas 1000 iterações considerando 4 tratamentos e 4 repetições
independentes cada -- portanto amostras de um tamanho total de 16
unidades -- advindas de distribuições normais com variância igual a 25.
Dessa forma, são satisfeitos os pressupostos da hipótese nula da ANOVA e
dos testes de comparações múltiplas: (1) independência, (2) normalidade
e (3) homocedasticidade.

Para as amostras dos tópicos abaixo, primeiramente é gerado um seed para
controlar a geração das 1000 seed únicos seguintes (cuja parte inteira
apenas é considerada), usadas na geração das amostras. Assim garante-se
a replicabilidade do experimento. Como todas as amostras são geradas
aleatoreamente sem qualquer dependência, considera-se que são
independentes. Por fim, cada
\(\text{amostra}_k; \, k \in \{1, 2, ..., 1000\}\) de tamanho 16 é
gerada com um \(\text{seed}_k\) correspondente.

\hypertarget{erro-tipo-i-em-comparauxe7uxf5es-muxfaltiplas}{%
\subsection{Erro Tipo I em comparações
múltiplas}\label{erro-tipo-i-em-comparauxe7uxf5es-muxfaltiplas}}

Para realizar os testes de comparações múltiplas de médias, foram
utilizadas as seguintes funções e seus testes correspondentes:

\begin{itemize}
\tightlist
\item
  Teste de Tukey - \texttt{TukeyHSD()};
\item
  Teste de Fisher - \texttt{pairwise.t.test()}, sem correção para
  \(\alpha\);
\item
  Teste de Fisher -
  \texttt{pairwise.t.test(...,\ p.adjust.method\ =\ "bonferroni")},
  utilizando a correção de Bonferroni para \(\alpha\).
\end{itemize}

Para o primeiro, foi observado 5.3\% de ocorrência de erro tipo I. Para
o segundo foi observado 194\% e para o terceiro 39\%.\\

\hypertarget{comparauxe7uxf5es-como-proteuxe7uxe3o-contra-erro-tipo-i}{%
\subsection{Comparações como proteção contra Erro Tipo
I}\label{comparauxe7uxf5es-como-proteuxe7uxe3o-contra-erro-tipo-i}}

Observa-se da simulação que em 57 casos houve erro do tipo 1
considerando \(\alpha = 0,05\), o que representa 5.7\% dos casos. Para
testar se um teste seguinte de comparações múltiplas auxiliaria em
reduzir a incidência de erro tipo I, realizou-se os mesmos testes do
tópico anterior apenas sobre as amostras em que houve esse erro de
acordo com a ANOVA. O ganho de precisao, ou redução do erro tipo I, é
exposto na tabela a seguir:

\begin{longtable}{lrr}
\toprule
Testes & ET1.dos.testes & Redução....\\
\midrule
\endfirsthead
\multicolumn{3}{@{}l}{\textit{(continued)}}\\
\toprule
Testes & ET1.dos.testes & Redução....\\
\midrule
\endhead

\endfoot
\bottomrule
\endlastfoot
\cellcolor{gray!15}{Tukey} & \cellcolor{gray!15}{46} & \cellcolor{gray!15}{1.1}\\
Fisher & 57 & 0.0\\
\cellcolor{gray!15}{Fisher (Bonferroni)} & \cellcolor{gray!15}{38} & \cellcolor{gray!15}{1.9}\\*
\end{longtable}

Nota-se portanto que, realizando os pós-testes de Tukey ou Fisher com
correção de Bonferroni, que controlam para esse tipo de erro, é possível
aumentar a precisão da análise em pelo menos 1\%. Contrariamente, o
teste de Fisher sem ajuste no p-valor não fornece qualquer melhoria na
análise, o que é esperado pois tipicamente há inflacionamento de erro
tipo I.

\hypertarget{anexo-1---cuxf3digo}{%
\section{Anexo 1 - código}\label{anexo-1---cuxf3digo}}

\begin{Shaded}
\begin{Highlighting}[]
\FunctionTok{library}\NormalTok{(kableExtra)}
\FunctionTok{library}\NormalTok{(broom)}
\FunctionTok{library}\NormalTok{(car)}
\FunctionTok{library}\NormalTok{(dplyr)}

\CommentTok{\# dados questao 1}
\NormalTok{tipo }\OtherTok{\textless{}{-}} \FunctionTok{factor}\NormalTok{(}\FunctionTok{c}\NormalTok{(}\StringTok{"I"}\NormalTok{, }\StringTok{"II"}\NormalTok{, }\StringTok{"III"}\NormalTok{))}
\NormalTok{tempo }\OtherTok{\textless{}{-}} \FunctionTok{c}\NormalTok{(}\DecValTok{19}\NormalTok{,}\DecValTok{20}\NormalTok{,}\DecValTok{16}\NormalTok{,}
           \DecValTok{22}\NormalTok{,}\DecValTok{21}\NormalTok{,}\DecValTok{15}\NormalTok{,}
           \DecValTok{20}\NormalTok{,}\DecValTok{33}\NormalTok{,}\DecValTok{18}\NormalTok{,}
           \DecValTok{18}\NormalTok{,}\DecValTok{27}\NormalTok{,}\DecValTok{26}\NormalTok{,}
           \DecValTok{25}\NormalTok{,}\DecValTok{40}\NormalTok{,}\DecValTok{17}\NormalTok{)}
\NormalTok{dados }\OtherTok{\textless{}{-}} \FunctionTok{data.frame}\NormalTok{(tipo, tempo)}

\CommentTok{\# tabela anova}
\NormalTok{tabela }\OtherTok{\textless{}{-}} \FunctionTok{aov}\NormalTok{(tempo }\SpecialCharTok{\textasciitilde{}}\NormalTok{ tipo, dados)}
\NormalTok{broom}\SpecialCharTok{::}\FunctionTok{tidy}\NormalTok{(tabela)}

\CommentTok{\# teste normalidade shapiro}
\FunctionTok{shapiro.test}\NormalTok{(tabela}\SpecialCharTok{$}\NormalTok{residuals) }\SpecialCharTok{\%\textgreater{}\%} 
    \FunctionTok{tidy}\NormalTok{()}

\CommentTok{\# teste homocedasticidade levene}
\NormalTok{car}\SpecialCharTok{::}\FunctionTok{leveneTest}\NormalTok{(tempo }\SpecialCharTok{\textasciitilde{}}\NormalTok{ tipo, dados) }\SpecialCharTok{\%\textgreater{}\%} 
    \FunctionTok{tidy}\NormalTok{() }

\CommentTok{\# teste Tukey}
\FunctionTok{TukeyHSD}\NormalTok{(tabela) }\SpecialCharTok{\%\textgreater{}\%} 
    \FunctionTok{tidy}\NormalTok{()}

\CommentTok{\# medias dos grupos}
\NormalTok{dados }\SpecialCharTok{\%\textgreater{}\%}
    \FunctionTok{group\_by}\NormalTok{(tipo)}\SpecialCharTok{\%\textgreater{}\%}
    \FunctionTok{summarise}\NormalTok{(}\AttributeTok{media =} \FunctionTok{mean}\NormalTok{(tempo))}

\CommentTok{\#IC}
\NormalTok{gl }\OtherTok{\textless{}{-}} \DecValTok{3}\SpecialCharTok{*}\DecValTok{5} \SpecialCharTok{{-}} \DecValTok{3} \CommentTok{\#(an {-} a)}
\NormalTok{alfa }\OtherTok{\textless{}{-}}\NormalTok{ (}\DecValTok{1}\FloatTok{{-}0.98}\NormalTok{)}\SpecialCharTok{/}\DecValTok{2}

\FunctionTok{qt}\NormalTok{(alfa, gl)}

\NormalTok{IC }\OtherTok{\textless{}{-}} \FunctionTok{qt}\NormalTok{(alfa, gl, }\AttributeTok{lower.tail =} \ConstantTok{FALSE}\NormalTok{)}\SpecialCharTok{*}\FunctionTok{sqrt}\NormalTok{(}\FloatTok{32.56667}\SpecialCharTok{/}\DecValTok{5}\NormalTok{)}
\FloatTok{18.4}\SpecialCharTok{+}\NormalTok{IC}
\FloatTok{18.4}\SpecialCharTok{{-}}\NormalTok{IC}

\CommentTok{\# questao 2 a {-}{-}{-}{-}{-}{-}{-}{-}{-}{-}{-}{-}{-}{-}{-}{-}{-}{-}{-}{-}{-}{-}{-}{-}{-}{-}{-}{-}{-}{-}{-}{-}{-}{-}{-}{-}{-}{-}{-}}
\CommentTok{\#seed inicial para gerar as seeds das amostras}
\FunctionTok{set.seed}\NormalTok{(}\DecValTok{12}\NormalTok{)}

\CommentTok{\#seeds para a geracao de amostras nas iteracoes}
\NormalTok{random }\OtherTok{\textless{}{-}} \FunctionTok{unique}\NormalTok{(}\FunctionTok{as.integer}\NormalTok{(}\FunctionTok{runif}\NormalTok{(}\DecValTok{1000}\NormalTok{, }\DecValTok{1}\NormalTok{, }\DecValTok{500000}\NormalTok{)))}

\CommentTok{\#um vetor que receberá todos os p{-}valor}
\NormalTok{resultados\_tukey }\OtherTok{\textless{}{-}} \DecValTok{0}
\NormalTok{resultados\_fisher }\OtherTok{\textless{}{-}} \DecValTok{0}
\NormalTok{resultados\_fisher\_bonferroni }\OtherTok{\textless{}{-}} \DecValTok{0}

\CommentTok{\#prepara os resultados de ANOVA para o bloco seguinte:}
\NormalTok{resultados\_aov }\OtherTok{\textless{}{-}} \FunctionTok{data.frame}\NormalTok{()}

\CommentTok{\#geracao das amostras e registro das ocorrencias}
\ControlFlowTok{for}\NormalTok{ (k }\ControlFlowTok{in} \DecValTok{1}\SpecialCharTok{:}\DecValTok{1000}\NormalTok{)\{ }\CommentTok{\#mil iteracoes}
  \FunctionTok{set.seed}\NormalTok{(random[k]) }\CommentTok{\#seleciona a seed correspondente}
\NormalTok{  amostra }\OtherTok{\textless{}{-}} \FunctionTok{data.frame}\NormalTok{( }
    \AttributeTok{trat =} \FunctionTok{factor}\NormalTok{(}\FunctionTok{c}\NormalTok{(}\StringTok{\textquotesingle{}I\textquotesingle{}}\NormalTok{, }\StringTok{\textquotesingle{}II\textquotesingle{}}\NormalTok{, }\StringTok{\textquotesingle{}III\textquotesingle{}}\NormalTok{, }\StringTok{\textquotesingle{}IV\textquotesingle{}}\NormalTok{)),}
    \AttributeTok{medidas =} \FunctionTok{rnorm}\NormalTok{(}\DecValTok{16}\NormalTok{, }\AttributeTok{sd =} \DecValTok{5}\NormalTok{)) }\CommentTok{\#gera amostras com o seed configurado}
  
  \CommentTok{\#vetor de pvalores do teste de comparacoes multiplas TUKEY}
\NormalTok{  pvalores }\OtherTok{\textless{}{-}} \FunctionTok{TukeyHSD}\NormalTok{(}\FunctionTok{aov}\NormalTok{(medidas }\SpecialCharTok{\textasciitilde{}}\NormalTok{ trat, amostra))}\SpecialCharTok{$}\NormalTok{trat[,}\DecValTok{4}\NormalTok{]}
  \ControlFlowTok{if}\NormalTok{(}\FunctionTok{sum}\NormalTok{(pvalores }\SpecialCharTok{\textless{}} \FloatTok{0.05}\NormalTok{, }\AttributeTok{na.rm =} \ConstantTok{TRUE}\NormalTok{) }\SpecialCharTok{\textgreater{}}\DecValTok{0}\NormalTok{)\{resultados\_tukey }\OtherTok{=}\NormalTok{ resultados\_tukey }\SpecialCharTok{+} \DecValTok{1}\NormalTok{\}}
  
  \CommentTok{\#vetor de pvalores do teste de comparacoes multiplas FISHER sem ajuste}
\NormalTok{  pvalores }\OtherTok{\textless{}{-}} \FunctionTok{pairwise.t.test}\NormalTok{(amostra}\SpecialCharTok{$}\NormalTok{medidas, amostra}\SpecialCharTok{$}\NormalTok{trat, }\AttributeTok{p.adjust.method =}\StringTok{"none"}\NormalTok{)}\SpecialCharTok{$}\NormalTok{p.value}
  \ControlFlowTok{if}\NormalTok{(}\FunctionTok{sum}\NormalTok{(pvalores }\SpecialCharTok{\textless{}} \FloatTok{0.05}\NormalTok{, }\AttributeTok{na.rm =} \ConstantTok{TRUE}\NormalTok{) }\SpecialCharTok{\textgreater{}}\DecValTok{0}\NormalTok{)\{resultados\_fisher }\OtherTok{=}\NormalTok{ resultados\_fisher }\SpecialCharTok{+} \DecValTok{1}\NormalTok{\}}
  
  \CommentTok{\#vetor de pvalores do teste de comparacoes multiplas FISHER BONFERRONI}
\NormalTok{  pvalores }\OtherTok{\textless{}{-}} \FunctionTok{pairwise.t.test}\NormalTok{(amostra}\SpecialCharTok{$}\NormalTok{medidas, amostra}\SpecialCharTok{$}\NormalTok{trat, }\AttributeTok{p.adjust.method =} \StringTok{"bonferroni"}\NormalTok{)}\SpecialCharTok{$}\NormalTok{p.value}
  \ControlFlowTok{if}\NormalTok{(}\FunctionTok{sum}\NormalTok{(pvalores }\SpecialCharTok{\textless{}} \FloatTok{0.05}\NormalTok{, }\AttributeTok{na.rm =} \ConstantTok{TRUE}\NormalTok{) }\SpecialCharTok{\textgreater{}}\DecValTok{0}\NormalTok{)\{resultados\_fisher\_bonferroni }\OtherTok{=}\NormalTok{ resultados\_fisher\_bonferroni }\SpecialCharTok{+} \DecValTok{1}\NormalTok{\}}
  
  \CommentTok{\# organizacao dos dados da anova}
\NormalTok{  temp\_aov }\OtherTok{\textless{}{-}} \FunctionTok{data.frame}\NormalTok{(}
    \AttributeTok{amostra =}\NormalTok{ k,}
    \AttributeTok{pvalor =}\NormalTok{ broom}\SpecialCharTok{::}\FunctionTok{tidy}\NormalTok{( }\CommentTok{\#registra p{-}valor na tabela}
    \FunctionTok{aov}\NormalTok{(medidas}\SpecialCharTok{\textasciitilde{}}\NormalTok{trat, }\AttributeTok{data =}\NormalTok{ amostra))}\SpecialCharTok{$}\NormalTok{p.value[}\DecValTok{1}\NormalTok{]}
\NormalTok{  )}
  
\NormalTok{  resultados\_aov }\OtherTok{\textless{}{-}} \FunctionTok{bind\_rows}\NormalTok{(resultados\_aov,temp\_aov)}
\NormalTok{\}}

\CommentTok{\# questao 2 b {-}{-}{-}{-}{-}{-}{-}{-}{-}{-}{-}{-}{-}{-}{-}{-}{-}{-}{-}{-}{-}{-}{-}{-}{-}{-}{-}{-}{-}{-}{-}{-}{-}{-}{-}{-}{-}{-}{-}{-}{-}{-}{-}{-}{-}{-}{-}}
\NormalTok{erro\_tipo1 }\OtherTok{\textless{}{-}} \FunctionTok{which}\NormalTok{(resultados\_aov}\SpecialCharTok{$}\NormalTok{pvalor }\SpecialCharTok{\textless{}=} \FloatTok{0.05}\NormalTok{)}
\NormalTok{n\_erro\_tipo\_I }\OtherTok{\textless{}{-}} \FunctionTok{length}\NormalTok{(erro\_tipo1)}

\CommentTok{\#um vetor que receberá todos os p{-}valor}
\NormalTok{resultados\_tukey2 }\OtherTok{\textless{}{-}} \DecValTok{0}
\NormalTok{resultados\_fisher2 }\OtherTok{\textless{}{-}} \DecValTok{0}
\NormalTok{resultados\_fisher\_bonferroni2 }\OtherTok{\textless{}{-}} \DecValTok{0}

\CommentTok{\#geracao das amostras e registro das ocorrencias}
\ControlFlowTok{for}\NormalTok{ (k }\ControlFlowTok{in}\NormalTok{ erro\_tipo1)\{ }\CommentTok{\#iteracoes com erro tipo 1}
  \FunctionTok{set.seed}\NormalTok{(random[k]) }\CommentTok{\#seleciona a seed correspondente}
\NormalTok{  amostra }\OtherTok{\textless{}{-}} \FunctionTok{data.frame}\NormalTok{( }
    \AttributeTok{trat =} \FunctionTok{factor}\NormalTok{(}\FunctionTok{c}\NormalTok{(}\StringTok{\textquotesingle{}I\textquotesingle{}}\NormalTok{, }\StringTok{\textquotesingle{}II\textquotesingle{}}\NormalTok{, }\StringTok{\textquotesingle{}III\textquotesingle{}}\NormalTok{, }\StringTok{\textquotesingle{}IV\textquotesingle{}}\NormalTok{)),}
    \AttributeTok{medidas =} \FunctionTok{rnorm}\NormalTok{(}\DecValTok{16}\NormalTok{, }\AttributeTok{sd =} \DecValTok{5}\NormalTok{)) }\CommentTok{\#gera amostras com o seed configurado}
  
  \CommentTok{\#vetor de pvalores do teste de comparacoes multiplas TUKEY}
\NormalTok{  pvalores }\OtherTok{\textless{}{-}} \FunctionTok{TukeyHSD}\NormalTok{(}\FunctionTok{aov}\NormalTok{(medidas }\SpecialCharTok{\textasciitilde{}}\NormalTok{ trat, amostra))}\SpecialCharTok{$}\NormalTok{trat[,}\DecValTok{4}\NormalTok{]}
  \ControlFlowTok{if}\NormalTok{(}\FunctionTok{sum}\NormalTok{(pvalores }\SpecialCharTok{\textless{}} \FloatTok{0.05}\NormalTok{, }\AttributeTok{na.rm =} \ConstantTok{TRUE}\NormalTok{) }\SpecialCharTok{\textgreater{}}\DecValTok{0}\NormalTok{)\{resultados\_tukey2 }\OtherTok{=}\NormalTok{ resultados\_tukey2 }\SpecialCharTok{+} \DecValTok{1}\NormalTok{\}}
  
  \CommentTok{\#vetor de pvalores do teste de comparacoes multiplas FISHER sem ajuste}
\NormalTok{  pvalores }\OtherTok{\textless{}{-}} \FunctionTok{pairwise.t.test}\NormalTok{(amostra}\SpecialCharTok{$}\NormalTok{medidas, amostra}\SpecialCharTok{$}\NormalTok{trat, }\AttributeTok{p.adjust.method =}\StringTok{"none"}\NormalTok{)}\SpecialCharTok{$}\NormalTok{p.value}
  \ControlFlowTok{if}\NormalTok{(}\FunctionTok{sum}\NormalTok{(pvalores }\SpecialCharTok{\textless{}} \FloatTok{0.05}\NormalTok{, }\AttributeTok{na.rm =} \ConstantTok{TRUE}\NormalTok{) }\SpecialCharTok{\textgreater{}}\DecValTok{0}\NormalTok{)\{resultados\_fisher2 }\OtherTok{=}\NormalTok{ resultados\_fisher2 }\SpecialCharTok{+} \DecValTok{1}\NormalTok{\}}
  
  \CommentTok{\#vetor de pvalores do teste de comparacoes multiplas FISHER BONFERRONI}
\NormalTok{  pvalores }\OtherTok{\textless{}{-}} \FunctionTok{pairwise.t.test}\NormalTok{(amostra}\SpecialCharTok{$}\NormalTok{medidas, amostra}\SpecialCharTok{$}\NormalTok{trat, }\AttributeTok{p.adjust.method =} \StringTok{"bonferroni"}\NormalTok{)}\SpecialCharTok{$}\NormalTok{p.value}
  \ControlFlowTok{if}\NormalTok{(}\FunctionTok{sum}\NormalTok{(pvalores }\SpecialCharTok{\textless{}} \FloatTok{0.05}\NormalTok{, }\AttributeTok{na.rm =} \ConstantTok{TRUE}\NormalTok{) }\SpecialCharTok{\textgreater{}}\DecValTok{0}\NormalTok{)\{resultados\_fisher\_bonferroni2 }\OtherTok{=}\NormalTok{ resultados\_fisher\_bonferroni2 }\SpecialCharTok{+} \DecValTok{1}\NormalTok{\}}
\NormalTok{\}}

\NormalTok{tabela\_erros }\OtherTok{\textless{}{-}} \FunctionTok{data.frame}\NormalTok{(}
  \AttributeTok{Testes =} \FunctionTok{c}\NormalTok{(}\StringTok{"Tukey"}\NormalTok{, }\StringTok{"Fisher"}\NormalTok{, }\StringTok{"Fisher (Bonferroni)"}\NormalTok{),}
  \StringTok{\textasciigrave{}}\AttributeTok{ET1 dos testes}\StringTok{\textasciigrave{}} \OtherTok{=} \FunctionTok{c}\NormalTok{(resultados\_tukey2, resultados\_fisher2, resultados\_fisher\_bonferroni2),}
  \StringTok{\textasciigrave{}}\AttributeTok{Redução (\%)}\StringTok{\textasciigrave{}} \OtherTok{=}\NormalTok{ (n\_erro\_tipo\_I}\SpecialCharTok{{-}}\FunctionTok{c}\NormalTok{(resultados\_tukey2, resultados\_fisher2, resultados\_fisher\_bonferroni2))}\SpecialCharTok{/}\DecValTok{10}
\NormalTok{)}
\end{Highlighting}
\end{Shaded}


\end{document}
