% Options for packages loaded elsewhere
\PassOptionsToPackage{unicode}{hyperref}
\PassOptionsToPackage{hyphens}{url}
%
\documentclass[
]{article}
\usepackage{amsmath,amssymb}
\usepackage{lmodern}
\usepackage{iftex}
\ifPDFTeX
  \usepackage[T1]{fontenc}
  \usepackage[utf8]{inputenc}
  \usepackage{textcomp} % provide euro and other symbols
\else % if luatex or xetex
  \usepackage{unicode-math}
  \defaultfontfeatures{Scale=MatchLowercase}
  \defaultfontfeatures[\rmfamily]{Ligatures=TeX,Scale=1}
\fi
% Use upquote if available, for straight quotes in verbatim environments
\IfFileExists{upquote.sty}{\usepackage{upquote}}{}
\IfFileExists{microtype.sty}{% use microtype if available
  \usepackage[]{microtype}
  \UseMicrotypeSet[protrusion]{basicmath} % disable protrusion for tt fonts
}{}
\makeatletter
\@ifundefined{KOMAClassName}{% if non-KOMA class
  \IfFileExists{parskip.sty}{%
    \usepackage{parskip}
  }{% else
    \setlength{\parindent}{0pt}
    \setlength{\parskip}{6pt plus 2pt minus 1pt}}
}{% if KOMA class
  \KOMAoptions{parskip=half}}
\makeatother
\usepackage{xcolor}
\IfFileExists{xurl.sty}{\usepackage{xurl}}{} % add URL line breaks if available
\IfFileExists{bookmark.sty}{\usepackage{bookmark}}{\usepackage{hyperref}}
\hypersetup{
  pdftitle={Experimentos Fatoriais},
  pdfauthor={César A. Galvão - 19/0011572},
  hidelinks,
  pdfcreator={LaTeX via pandoc}}
\urlstyle{same} % disable monospaced font for URLs
\usepackage[margin=1in]{geometry}
\usepackage{graphicx}
\makeatletter
\def\maxwidth{\ifdim\Gin@nat@width>\linewidth\linewidth\else\Gin@nat@width\fi}
\def\maxheight{\ifdim\Gin@nat@height>\textheight\textheight\else\Gin@nat@height\fi}
\makeatother
% Scale images if necessary, so that they will not overflow the page
% margins by default, and it is still possible to overwrite the defaults
% using explicit options in \includegraphics[width, height, ...]{}
\setkeys{Gin}{width=\maxwidth,height=\maxheight,keepaspectratio}
% Set default figure placement to htbp
\makeatletter
\def\fps@figure{htbp}
\makeatother
\setlength{\emergencystretch}{3em} % prevent overfull lines
\providecommand{\tightlist}{%
  \setlength{\itemsep}{0pt}\setlength{\parskip}{0pt}}
\setcounter{secnumdepth}{-\maxdimen} % remove section numbering
\usepackage{helvet} \renewcommand\familydefault{\sfdefault}
\usepackage{booktabs}
\usepackage{longtable}
\usepackage{array}
\usepackage{multirow}
\usepackage{wrapfig}
\usepackage{float}
\usepackage{colortbl}
\usepackage{pdflscape}
\usepackage{tabu}
\usepackage{threeparttable}
\usepackage{threeparttablex}
\usepackage[normalem]{ulem}
\usepackage{makecell}
\usepackage{xcolor}
\ifLuaTeX
  \usepackage{selnolig}  % disable illegal ligatures
\fi

\title{Experimentos Fatoriais}
\author{César A. Galvão - 19/0011572}
\date{2022-08-19}

\begin{document}
\maketitle

\newpage{}

{
\setcounter{tocdepth}{2}
\tableofcontents
}
\let\oldsection\section
\renewcommand\section{\clearpage\oldsection}

\hypertarget{section}{%
\section{}\label{section}}

\hypertarget{modelo-e-anova}{%
\subsection{Modelo e ANOVA}\label{modelo-e-anova}}

É utilizado o modelo de experimentos fatoriais, representado por:

\begin{align*}
  y_{ijk} = \mu + \tau_i + \beta_j + \left( \tau\beta \right)_{ij} + e_{ijk}, \quad i = 1, 2,..., a; \quad j = 1, 2,..., b \quad k = 1, 2,..., n
\end{align*}

em que \(\mu\) é a média geral, \(\tau_i\) é o efeito do fator
\textbf{vidro}, \(\beta_j\) é o efeito do fator \textbf{fósforo},
\((\tau\beta)_{ij}\) é o efeito de interação entre os dois fatores e
\(e_{ijk}\) é o desvio do elemento. Portanto, existem
\(a \cdot b = 3 \cdot 2 = 6\) tratamentos possíveis para este
experimento.

\begin{longtable}{cccccc}
\toprule
term & df & sumsq & meansq & statistic & p.value\\
\midrule
\endfirsthead
\multicolumn{6}{@{}l}{\textit{(continued)}}\\
\toprule
term & df & sumsq & meansq & statistic & p.value\\
\midrule
\endhead

\endfoot
\bottomrule
\endlastfoot
\cellcolor{gray!15}{phosphor} & \cellcolor{gray!15}{2} & \cellcolor{gray!15}{933.3333} & \cellcolor{gray!15}{466.6667} & \cellcolor{gray!15}{8.8421} & \cellcolor{gray!15}{0.0044}\\
glass & 1 & 14450.0000 & 14450.0000 & 273.7895 & 0.0000\\
\cellcolor{gray!15}{phosphor:glass} & \cellcolor{gray!15}{2} & \cellcolor{gray!15}{133.3333} & \cellcolor{gray!15}{66.6667} & \cellcolor{gray!15}{1.2632} & \cellcolor{gray!15}{0.3178}\\
Residuals & 12 & 633.3333 & 52.7778 & NA & NA\\*
\end{longtable}

Pela tabela de ANOVA, os efeitos de ambos os fatores do experimento são
significativos considerando mesmo \(\alpha = 0,01\). No entanto
rejeita-se a hipótese de existência de interação entre os fatores. Ou
seja, pode-se considerar os efeitos do tipo de vidro e do tipo de
fósforo independentes.

\hypertarget{estimadores}{%
\subsection{Estimadores}\label{estimadores}}

\begin{longtable}{cc}
\toprule
$\mu$ & $\sigma^2$\\
\midrule
\endfirsthead
\multicolumn{2}{@{}l}{\textit{(continued)}}\\
\toprule
$\mu$ & $\sigma^2$\\
\midrule
\endhead

\endfoot
\bottomrule
\endlastfoot
\cellcolor{gray!15}{263.333} & \cellcolor{gray!15}{52.778}\\*
\end{longtable}

\begin{longtable}{cc}
\toprule
$\tau_1$ & $\tau_2$\\
\midrule
\endfirsthead
\multicolumn{2}{@{}l}{\textit{(continued)}}\\
\toprule
$\tau_1$ & $\tau_2$\\
\midrule
\endhead

\endfoot
\bottomrule
\endlastfoot
\cellcolor{gray!15}{28.333} & \cellcolor{gray!15}{-28.333}\\*
\end{longtable}

\begin{longtable}{ccc}
\toprule
$\beta_1$ & $\beta_2$ & $\beta_3$\\
\midrule
\endfirsthead
\multicolumn{3}{@{}l}{\textit{(continued)}}\\
\toprule
$\beta_1$ & $\beta_2$ & $\beta_3$\\
\midrule
\endhead

\endfoot
\bottomrule
\endlastfoot
\cellcolor{gray!15}{-3.333} & \cellcolor{gray!15}{10} & \cellcolor{gray!15}{-6.667}\\*
\end{longtable}

\begin{longtable}{cccccc}
\toprule
$\tau_1\beta_1$ & $\tau_1\beta_2$ & $\tau_1\beta_3$ & $\tau_2\beta_1$ & $\tau_2\beta_2$ & $\tau_2\beta_3$\\
\midrule
\endfirsthead
\multicolumn{6}{@{}l}{\textit{(continued)}}\\
\toprule
$\tau_1\beta_1$ & $\tau_1\beta_2$ & $\tau_1\beta_3$ & $\tau_2\beta_1$ & $\tau_2\beta_2$ & $\tau_2\beta_3$\\
\midrule
\endhead

\endfoot
\bottomrule
\endlastfoot
\cellcolor{gray!15}{21.667} & \cellcolor{gray!15}{38.333} & \cellcolor{gray!15}{25} & \cellcolor{gray!15}{-28.333} & \cellcolor{gray!15}{-18.333} & \cellcolor{gray!15}{-38.333}\\*
\end{longtable}

parametro de nao centralidade: E(SQA/sigma\^{}2), SQA = (a-1)QMA

\end{document}
