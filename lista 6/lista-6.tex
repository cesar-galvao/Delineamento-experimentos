% Options for packages loaded elsewhere
\PassOptionsToPackage{unicode}{hyperref}
\PassOptionsToPackage{hyphens}{url}
%
\documentclass[
]{article}
\usepackage{amsmath,amssymb}
\usepackage{lmodern}
\usepackage{iftex}
\ifPDFTeX
  \usepackage[T1]{fontenc}
  \usepackage[utf8]{inputenc}
  \usepackage{textcomp} % provide euro and other symbols
\else % if luatex or xetex
  \usepackage{unicode-math}
  \defaultfontfeatures{Scale=MatchLowercase}
  \defaultfontfeatures[\rmfamily]{Ligatures=TeX,Scale=1}
\fi
% Use upquote if available, for straight quotes in verbatim environments
\IfFileExists{upquote.sty}{\usepackage{upquote}}{}
\IfFileExists{microtype.sty}{% use microtype if available
  \usepackage[]{microtype}
  \UseMicrotypeSet[protrusion]{basicmath} % disable protrusion for tt fonts
}{}
\makeatletter
\@ifundefined{KOMAClassName}{% if non-KOMA class
  \IfFileExists{parskip.sty}{%
    \usepackage{parskip}
  }{% else
    \setlength{\parindent}{0pt}
    \setlength{\parskip}{6pt plus 2pt minus 1pt}}
}{% if KOMA class
  \KOMAoptions{parskip=half}}
\makeatother
\usepackage{xcolor}
\IfFileExists{xurl.sty}{\usepackage{xurl}}{} % add URL line breaks if available
\IfFileExists{bookmark.sty}{\usepackage{bookmark}}{\usepackage{hyperref}}
\hypersetup{
  pdftitle={Exercício de laboratório 6},
  pdfauthor={César A. Galvão - 19/0011572},
  hidelinks,
  pdfcreator={LaTeX via pandoc}}
\urlstyle{same} % disable monospaced font for URLs
\usepackage[margin=1in]{geometry}
\usepackage{graphicx}
\makeatletter
\def\maxwidth{\ifdim\Gin@nat@width>\linewidth\linewidth\else\Gin@nat@width\fi}
\def\maxheight{\ifdim\Gin@nat@height>\textheight\textheight\else\Gin@nat@height\fi}
\makeatother
% Scale images if necessary, so that they will not overflow the page
% margins by default, and it is still possible to overwrite the defaults
% using explicit options in \includegraphics[width, height, ...]{}
\setkeys{Gin}{width=\maxwidth,height=\maxheight,keepaspectratio}
% Set default figure placement to htbp
\makeatletter
\def\fps@figure{htbp}
\makeatother
\setlength{\emergencystretch}{3em} % prevent overfull lines
\providecommand{\tightlist}{%
  \setlength{\itemsep}{0pt}\setlength{\parskip}{0pt}}
\setcounter{secnumdepth}{5}
\usepackage{helvet} \renewcommand\familydefault{\sfdefault}
\usepackage{booktabs}
\usepackage{longtable}
\usepackage{array}
\usepackage{multirow}
\usepackage{wrapfig}
\usepackage{float}
\usepackage{colortbl}
\usepackage{pdflscape}
\usepackage{tabu}
\usepackage{threeparttable}
\usepackage{threeparttablex}
\usepackage[normalem]{ulem}
\usepackage{makecell}
\usepackage{xcolor}
\ifLuaTeX
  \usepackage{selnolig}  % disable illegal ligatures
\fi

\title{Exercício de laboratório 6}
\usepackage{etoolbox}
\makeatletter
\providecommand{\subtitle}[1]{% add subtitle to \maketitle
  \apptocmd{\@title}{\par {\large #1 \par}}{}{}
}
\makeatother
\subtitle{Quadrado latino}
\author{César A. Galvão - 19/0011572}
\date{2022-08-11}

\begin{document}
\maketitle

\newpage{}

{
\setcounter{tocdepth}{2}
\tableofcontents
}
\let\oldsection\section
\renewcommand\section{\clearpage\oldsection}

\hypertarget{questuxe3o-1}{%
\section{Questão 1}\label{questuxe3o-1}}

\hypertarget{modelo}{%
\subsection{Modelo}\label{modelo}}

Para o experimento de quadrado latino utiliza-se o seguinte modelo de
efeitos:

\begin{align}
  y_{ijk} = \mu + \alpha_i + \tau_j + \beta_k + \varepsilon_{ijk}, 
  \begin{cases} 
  i, k = 1, 2, 3 \\
  j = A, B, C
  \end{cases}
\end{align}

onde \(y_{ijk}\) é a observação na i-ésima linha, na k-ésima coluna para
o j-ésimo tratamento. \(\mu\) é a média total, \(\alpha_i\) é o efeito
da i-ésima linha, \(\tau_j\) é o efeito da j-ésimo tratamento,
\(\beta_k\) é o efeito da k-ésima coluna e \(\varepsilon_{ijk}\) é o
erro aleatório. O modelo é completamente aditivo; nesse sentido, não há
interação entre linhas, colunas e tratamentos. Como só há uma observação
em cada célula, apenas dois dos três subscritos i, j e k são necessários
para denotar uma observação em particular.

Dessa forma, testa-se a igualdade dos efeitos de tratamento ou, em
outras palavras, se o efeito dos tratamentos é igual a zero.

Desse modo, as hipóteses são:

\begin{align}
  \begin{cases}
    H_0: \mu_1 = ... = \mu_a\\
    H_1: \exists \mu_i \neq \mu_j, \, i \neq j.
  \end{cases}
\end{align}

ou

\begin{align}
  \begin{cases}
    H_0: \tau_1 = ... = \tau_a = 0, \quad \text{(O efeito de tratamento é nulo)}\\
    H_1: \exists \tau_i \neq 0
  \end{cases}
\end{align}

A análise de variância consiste em particionar a soma de quadrados total
das observações, nos componentes para linhas, colunas, tratamentos e
erro:

\begin{equation}
 SQ_\text{Tot} = SQ_\text{{linhas}} + SQ_\text{{Colunas}} + + SQ_\text{{Trat}} + SQ_\text{Res}
\end{equation}

\noindent com os respectivos graus de liberdade:

\begin{align}
    (p^2-1) = (p-1) + (p-1) + (p-1) + (p-2)(p-1)
 \end{align}

Assumindo que \(\varepsilon_{ijk}\) segue uma distribuição
\(N(0,\sigma^2)\), cada soma de quadrados do lado direito da equação
apresentada acima é uma variável qui-quadrado independente quando
dividida por \(\sigma^2\). A estatística de teste apropriada para testar
as diferenças entre os tratamentos é

\begin{equation}
    F_0 = \frac{QM_{tratamentos}}{QMRES} \overset{H_0}{\sim} F_{(p-1),(p-2)(p-1)}
\end{equation}

\hypertarget{tabela-anova}{%
\subsection{Tabela ANOVA}\label{tabela-anova}}

A tabela de análise de variância é apresentada a seguir, na qual é
possível observar que todos os efeitos do modelo são significativos.
Isto significa que, numa futura repetição do experimento, recomenda-se
repetir a estrutura de casualização.

\begin{longtable}{cccccc}
\toprule
term & df & sumsq & meansq & statistic & p.value\\
\midrule
\endfirsthead
\multicolumn{6}{@{}l}{\textit{(continued)}}\\
\toprule
term & df & sumsq & meansq & statistic & p.value\\
\midrule
\endhead

\endfoot
\bottomrule
\endlastfoot
\cellcolor{gray!15}{alfa} & \cellcolor{gray!15}{2} & \cellcolor{gray!15}{928005.556} & \cellcolor{gray!15}{464002.778} & \cellcolor{gray!15}{103.2315} & \cellcolor{gray!15}{0.0096}\\
beta & 2 & 261114.889 & 130557.444 & 29.0465 & 0.0333\\
\cellcolor{gray!15}{trat} & \cellcolor{gray!15}{2} & \cellcolor{gray!15}{608890.889} & \cellcolor{gray!15}{304445.444} & \cellcolor{gray!15}{67.7331} & \cellcolor{gray!15}{0.0145}\\
Residuals & 2 & 8989.556 & 4494.778 & NA & NA\\*
\end{longtable}

\hypertarget{estimadores}{%
\subsection{Estimadores}\label{estimadores}}

De acordo com o modelo, os seguintes são os estimadores para média,
variância:

\begin{longtable}{cc}
\toprule
$\mu$ & $\sigma^2$\\
\midrule
\endfirsthead
\multicolumn{2}{@{}l}{\textit{(continued)}}\\
\toprule
$\mu$ & $\sigma^2$\\
\midrule
\endhead

\endfoot
\bottomrule
\endlastfoot
\cellcolor{gray!15}{1706.11} & \cellcolor{gray!15}{4494.78}\\*
\end{longtable}

\begin{longtable}{ccc}
\toprule
$\tau_1$ & $\tau_2$ & $\tau_3$\\
\midrule
\endfirsthead
\multicolumn{3}{@{}l}{\textit{(continued)}}\\
\toprule
$\tau_1$ & $\tau_2$ & $\tau_3$\\
\midrule
\endhead

\endfoot
\bottomrule
\endlastfoot
\cellcolor{gray!15}{-224.78} & \cellcolor{gray!15}{-139.78} & \cellcolor{gray!15}{364.56}\\*
\end{longtable}

\begin{longtable}{ccc}
\toprule
$\beta_1$ & $\beta_2$ & $\beta_3$\\
\midrule
\endfirsthead
\multicolumn{3}{@{}l}{\textit{(continued)}}\\
\toprule
$\beta_1$ & $\beta_2$ & $\beta_3$\\
\midrule
\endhead

\endfoot
\bottomrule
\endlastfoot
\cellcolor{gray!15}{224.56} & \cellcolor{gray!15}{-36.78} & \cellcolor{gray!15}{-187.78}\\*
\end{longtable}

\begin{longtable}{ccc}
\toprule
$\alpha_1$ & $\alpha_2$ & $\alpha_3$\\
\midrule
\endfirsthead
\multicolumn{3}{@{}l}{\textit{(continued)}}\\
\toprule
$\alpha_1$ & $\alpha_2$ & $\alpha_3$\\
\midrule
\endhead

\endfoot
\bottomrule
\endlastfoot
\cellcolor{gray!15}{50.56} & \cellcolor{gray!15}{-416.11} & \cellcolor{gray!15}{365.56}\\*
\end{longtable}

A seguir, testa-se os pressupostos de normalidade e homocedasticidade
para a utilização da ANOVA como um teste adequado:

\begin{longtable}{ccc}
\toprule
statistic & p.value & method\\
\midrule
\endfirsthead
\multicolumn{3}{@{}l}{\textit{(continued)}}\\
\toprule
statistic & p.value & method\\
\midrule
\endhead

\endfoot
\bottomrule
\endlastfoot
\cellcolor{gray!15}{0.7863} & \cellcolor{gray!15}{0.0142} & \cellcolor{gray!15}{Shapiro-Wilk normality test}\\*
\end{longtable}

Pelo teste Shapiro, rejeita-se normalidade. Trata-se de indicativo de
que um teste não-paramétrico seria mais adequado para avaliar as
distinções entre tratamentos.

Por fim, verifica-se que os dados são homocedásticos, dados os p-valores
dos testes Levene aplicados a seguir sobre os resíduos do modelo de
análise de variância.

\begin{longtable}{ccccc}
\toprule
statistic & p.value & df & df.residual & fonte\\
\midrule
\endfirsthead
\multicolumn{5}{@{}l}{\textit{(continued)}}\\
\toprule
statistic & p.value & df & df.residual & fonte\\
\midrule
\endhead

\endfoot
\bottomrule
\endlastfoot
\cellcolor{gray!15}{0} & \cellcolor{gray!15}{1} & \cellcolor{gray!15}{2} & \cellcolor{gray!15}{6} & \cellcolor{gray!15}{tratamento}\\
0 & 1 & 2 & 6 & bloco\\
\cellcolor{gray!15}{0} & \cellcolor{gray!15}{1} & \cellcolor{gray!15}{2} & \cellcolor{gray!15}{6} & \cellcolor{gray!15}{linha}\\*
\end{longtable}

\hypertarget{questuxe3o-2}{%
\section{Questão 2}\label{questuxe3o-2}}

\begin{verbatim}
##                    Df  Sum Sq Mean Sq F value   Pr(>F)    
## period2             2  727534  363767   1.773  0.19551    
## trat                2   80264   40132   0.196  0.82390    
## repeticao           3 8603312 2867771  13.977 3.84e-05 ***
## repeticao:subject2  8 7750447  968806   4.722  0.00229 ** 
## Residuals          20 4103618  205181                     
## ---
## Signif. codes:  0 '***' 0.001 '**' 0.01 '*' 0.05 '.' 0.1 ' ' 1
\end{verbatim}

\begin{verbatim}
## 
##  Shapiro-Wilk normality test
## 
## data:  tabela2$residuals
## W = 0.98396, p-value = 0.8689
\end{verbatim}

\begin{verbatim}
## Levene's Test for Homogeneity of Variance (center = median)
##       Df F value Pr(>F)
## group 11  0.7485 0.6841
##       24
\end{verbatim}

\begin{verbatim}
## Levene's Test for Homogeneity of Variance (center = median)
##       Df F value Pr(>F)
## group  2   0.304 0.7399
##       33
\end{verbatim}

\begin{verbatim}
## Levene's Test for Homogeneity of Variance (center = median)
##       Df F value Pr(>F)
## group  2  1.2776 0.2921
##       33
\end{verbatim}

\end{document}
