% Options for packages loaded elsewhere
\PassOptionsToPackage{unicode}{hyperref}
\PassOptionsToPackage{hyphens}{url}
%
\documentclass[
]{article}
\usepackage{amsmath,amssymb}
\usepackage{lmodern}
\usepackage{iftex}
\ifPDFTeX
  \usepackage[T1]{fontenc}
  \usepackage[utf8]{inputenc}
  \usepackage{textcomp} % provide euro and other symbols
\else % if luatex or xetex
  \usepackage{unicode-math}
  \defaultfontfeatures{Scale=MatchLowercase}
  \defaultfontfeatures[\rmfamily]{Ligatures=TeX,Scale=1}
\fi
% Use upquote if available, for straight quotes in verbatim environments
\IfFileExists{upquote.sty}{\usepackage{upquote}}{}
\IfFileExists{microtype.sty}{% use microtype if available
  \usepackage[]{microtype}
  \UseMicrotypeSet[protrusion]{basicmath} % disable protrusion for tt fonts
}{}
\makeatletter
\@ifundefined{KOMAClassName}{% if non-KOMA class
  \IfFileExists{parskip.sty}{%
    \usepackage{parskip}
  }{% else
    \setlength{\parindent}{0pt}
    \setlength{\parskip}{6pt plus 2pt minus 1pt}}
}{% if KOMA class
  \KOMAoptions{parskip=half}}
\makeatother
\usepackage{xcolor}
\IfFileExists{xurl.sty}{\usepackage{xurl}}{} % add URL line breaks if available
\IfFileExists{bookmark.sty}{\usepackage{bookmark}}{\usepackage{hyperref}}
\hypersetup{
  pdftitle={Prova 2},
  pdfauthor={César A. Galvão - 19/0011572},
  hidelinks,
  pdfcreator={LaTeX via pandoc}}
\urlstyle{same} % disable monospaced font for URLs
\usepackage[margin=1in]{geometry}
\usepackage{graphicx}
\makeatletter
\def\maxwidth{\ifdim\Gin@nat@width>\linewidth\linewidth\else\Gin@nat@width\fi}
\def\maxheight{\ifdim\Gin@nat@height>\textheight\textheight\else\Gin@nat@height\fi}
\makeatother
% Scale images if necessary, so that they will not overflow the page
% margins by default, and it is still possible to overwrite the defaults
% using explicit options in \includegraphics[width, height, ...]{}
\setkeys{Gin}{width=\maxwidth,height=\maxheight,keepaspectratio}
% Set default figure placement to htbp
\makeatletter
\def\fps@figure{htbp}
\makeatother
\setlength{\emergencystretch}{3em} % prevent overfull lines
\providecommand{\tightlist}{%
  \setlength{\itemsep}{0pt}\setlength{\parskip}{0pt}}
\setcounter{secnumdepth}{-\maxdimen} % remove section numbering
\usepackage{helvet} \renewcommand\familydefault{\sfdefault}
\usepackage{booktabs}
\usepackage{longtable}
\usepackage{array}
\usepackage{multirow}
\usepackage{wrapfig}
\usepackage{float}
\usepackage{colortbl}
\usepackage{pdflscape}
\usepackage{tabu}
\usepackage{threeparttable}
\usepackage{threeparttablex}
\usepackage[normalem]{ulem}
\usepackage{makecell}
\usepackage{xcolor}
\ifLuaTeX
  \usepackage{selnolig}  % disable illegal ligatures
\fi

\title{Prova 2}
\author{César A. Galvão - 19/0011572}
\date{2022-08-12}

\begin{document}
\maketitle

\newpage{}

{
\setcounter{tocdepth}{2}
\tableofcontents
}
\let\oldsection\section
\renewcommand\section{\clearpage\oldsection}

\hypertarget{questuxe3o-1}{%
\section{Questão 1}\label{questuxe3o-1}}

\begin{longtable}{ccccc}
\toprule
Tratamento & Bloco 1 & Bloco 2 & Bloco3 & Bloco 4\\
\midrule
\endfirsthead
\multicolumn{5}{@{}l}{\textit{(continued)}}\\
\toprule
Tratamento & Bloco 1 & Bloco 2 & Bloco3 & Bloco 4\\
\midrule
\endhead

\endfoot
\bottomrule
\endlastfoot
\cellcolor{gray!15}{1} & \cellcolor{gray!15}{105.17} & \cellcolor{gray!15}{102.21} & \cellcolor{gray!15}{99.43} & \cellcolor{gray!15}{107.74}\\
2 & 97.42 & 89.36 & 90.16 & 100.04\\
\cellcolor{gray!15}{3} & \cellcolor{gray!15}{100.78} & \cellcolor{gray!15}{99.26} & \cellcolor{gray!15}{96.77} & \cellcolor{gray!15}{102.50}\\
4 & 102.09 & 99.45 & 102.63 & 107.63\\*
\end{longtable}

É utilizado o RCBD, \emph{randomized complete block design},
representado por:

\begin{align*}
  y_{ij} = \mu + \tau_i + \beta_j + e_{ij}, \quad i = 1, 2,..., a; \quad j = 1, 2,..., n
\end{align*}

em que \(\mu\) é a média geral, \(\tau_i\) é a média ou efeito dos
grupos -- cada químico sendo considerado um tratamento --, \(\beta_j\) é
o bloco e \(e_{ij}\) é o desvio do elemento. Os grupos são indexados por
\(i\) e os blocos indexados por \(j\).

As hipóteses do teste são as seguintes: \begin{align*}
  \begin{cases}
    H_0: \tau_1 = ... = \tau_a = 0, \quad \text{(O efeito de tratamento é nulo)}\\
    H_1: \exists \tau_i \neq 0
  \end{cases}
\end{align*}

que equivale dizer

\begin{align*}
  \begin{cases}
    H_0: \mu_1 = ... = \mu_a\\
    H_1: \exists \mu_i \neq \mu_j, \, i \neq j.
  \end{cases}
\end{align*}

Mesmo que o interesse do estudo não seja sobre o efeito dos blocos, é
interessante testá-los para avaliar se é necessário manter a estrutura
de blocos e futuras replicações do experimento.

Apresenta-se inicialmente a tabela de ANOVA, na qual é possível observar
níveis de significância suficientes para se considerar tanto o efeito de
blocos quanto o efeito de tratamentos como significativos. Isso quer
dizer que de fato a blocagem teve efeito sobre os resultados do
experimento e que há pelo menos um tratamento que difere dos demais.

\begin{longtable}{cccccc}
\toprule
term & df & sumsq & meansq & statistic & p.value\\
\midrule
\endfirsthead
\multicolumn{6}{@{}l}{\textit{(continued)}}\\
\toprule
term & df & sumsq & meansq & statistic & p.value\\
\midrule
\endhead

\endfoot
\bottomrule
\endlastfoot
\cellcolor{gray!15}{blocos} & \cellcolor{gray!15}{3} & \cellcolor{gray!15}{141.1339} & \cellcolor{gray!15}{47.04465} & \cellcolor{gray!15}{12.19711} & \cellcolor{gray!15}{0.00160}\\
tratamentos & 3 & 219.8991 & 73.29972 & 19.00417 & 0.00031\\
\cellcolor{gray!15}{Residuals} & \cellcolor{gray!15}{9} & \cellcolor{gray!15}{34.7133} & \cellcolor{gray!15}{3.85703} & \cellcolor{gray!15}{NA} & \cellcolor{gray!15}{NA}\\*
\end{longtable}

Procede-se portante para o teste diagnóstico da análise de variância.
Especificamente, o teste para normalidade dos resíduos, conduzido
utilizando o teste de Shapiro, apresenta p-valor de 0.62. Ou seja, não
se rejeita a hipótese de normalidade dos dados. Além disso, o teste de
Levene para homocedasticidade apresenta, para blocos e tratamentos,
p-valores de 0.91 e 0.17 respectivamente, de modo que pode-se considerar
a homocedasticidade da amostra em ambas as dimensões.

Ainda, realiza-se teste de aditividade, para o qual a hipótese nula é de
que o experimento é completamente aditivo. Obtem-se p-valor 0.098, de
modo que não se rejeita a aditividade considerando \(\alpha = 0.05\).

Apresenta-se os parâmetros estimados a seguir:

\begin{longtable}{cc}
\toprule
$\mu$ & $\sigma^2$\\
\midrule
\endfirsthead
\multicolumn{2}{@{}l}{\textit{(continued)}}\\
\toprule
$\mu$ & $\sigma^2$\\
\midrule
\endhead

\endfoot
\bottomrule
\endlastfoot
\cellcolor{gray!15}{100.165} & \cellcolor{gray!15}{3.857}\\*
\end{longtable}

\begin{longtable}{cccc}
\toprule
$\tau_1$ & $\tau_2$ & $\tau_3$ & $\tau_4$\\
\midrule
\endfirsthead
\multicolumn{4}{@{}l}{\textit{(continued)}}\\
\toprule
$\tau_1$ & $\tau_2$ & $\tau_3$ & $\tau_4$\\
\midrule
\endhead

\endfoot
\bottomrule
\endlastfoot
\cellcolor{gray!15}{3.472} & \cellcolor{gray!15}{-5.92} & \cellcolor{gray!15}{-0.338} & \cellcolor{gray!15}{2.785}\\*
\end{longtable}

\begin{longtable}{cccc}
\toprule
$\beta_1$ & $\beta_2$ & $\beta_3$ & $\beta_4$\\
\midrule
\endfirsthead
\multicolumn{4}{@{}l}{\textit{(continued)}}\\
\toprule
$\beta_1$ & $\beta_2$ & $\beta_3$ & $\beta_4$\\
\midrule
\endhead

\endfoot
\bottomrule
\endlastfoot
\cellcolor{gray!15}{1.2} & \cellcolor{gray!15}{-2.595} & \cellcolor{gray!15}{-2.918} & \cellcolor{gray!15}{4.312}\\*
\end{longtable}

Por fim, realiza-se teste de Tukey para avaliar quais tratamentos
diferem entre si. Considerando significância de 0.05, não haveria
diferença apenas entre o tratamento 4 e os tratamentos 1 e 3.

\begin{longtable}{lcccc}
\toprule
  & diff & lwr & upr & p adj\\
\midrule
\endfirsthead
\multicolumn{5}{@{}l}{\textit{(continued)}}\\
\toprule
  & diff & lwr & upr & p adj\\
\midrule
\endhead

\endfoot
\bottomrule
\endlastfoot
\cellcolor{gray!15}{2-1} & \cellcolor{gray!15}{-9.39} & \cellcolor{gray!15}{-13.73} & \cellcolor{gray!15}{-5.06} & \cellcolor{gray!15}{0.00}\\
3-1 & -3.81 & -8.15 & 0.53 & 0.09\\
\cellcolor{gray!15}{4-1} & \cellcolor{gray!15}{-0.69} & \cellcolor{gray!15}{-5.02} & \cellcolor{gray!15}{3.65} & \cellcolor{gray!15}{0.96}\\
3-2 & 5.58 & 1.25 & 9.92 & 0.01\\
\cellcolor{gray!15}{4-2} & \cellcolor{gray!15}{8.70} & \cellcolor{gray!15}{4.37} & \cellcolor{gray!15}{13.04} & \cellcolor{gray!15}{0.00}\\
4-3 & 3.12 & -1.21 & 7.46 & 0.18\\*
\end{longtable}

O erro tipo II deste modelo é calculado com o seguite parâmetro de não
centralidade:

\begin{align}
  NCP = \phi^2 &= n \cdot \frac{\sum\limits_{i = 1}^{4} \tau_i^2}{\sigma^2} \quad \tau_i = \{ -2, 0, 0, 2\} \\
  &= 4 \cdot \frac{8}{3.857}\\
  &= 8.296
\end{align}

Considera-se ainda \((a-1)(b-1) = 3 \cdot 3 = 9\) graus de liberdade
para o denominador e \(a-1 = 3\) graus de liberdade para o numerador da
estatística F.

Obtém-se uma probabilidade de erro tipo II de 0.525

\hypertarget{questuxe3o-2}{%
\section{Questão 2}\label{questuxe3o-2}}

Na simulação exposta, estão sendo comparadas as ocorrências de erro tipo
II para os dois modelos, considerando ou não blocos, quando de fato não
existe diferença entre blocos. Pela forma como a simulação é desenhada,
entende-se que a blocagem não deve ser importante para a replicação do
experimento (o vetor de efeitos de bloco é nulo) e, sendo utilizada no
modelo, reduz-se os graus de liberdade do resíduo e consequentemente
aumenta-se o QMRES, que é o denominador da estatística de teste. De
fato, observa-se pelas probabilidades encontradas que o segundo modelo
apresenta uma maior probabilidade de erro tipo II, conforme esperado.

\hypertarget{questuxe3o-3}{%
\section{Questão 3}\label{questuxe3o-3}}

O teste de Friedman considera a média dos ranks dentro de cada bloco,
\(k\) tratamentos e \(n\) unidades dentro de cada tratamento. A
estatística de teste é a \(Q\) seguinte, que segue uma distribuição
\(\chi^2_{(k-1)}\):

\begin{align}
  Q = \frac{12n}{k(k+1)} \sum\limits^k_{j = 1}\left( \bar{r}_{.j} - \frac{k+1}{2} \right)^2
\end{align}

A seguir estão expostas duas tabelas: uma com a maior variabilidade
possível dentro dos blocos mas uniforme entre blocos, outra sem
variabilidade de rank dentro dos blocos, mas maior variabilidade entre
blocos.

É possível observar que a tabela cujos valores apresentam estatística de
teste igual a doze possui pvalor muito baixo e de fato avaliam os
tratamentos como consistentemente posicionados de forma ranqueada entre
os blocos. Em contrapartida, a tabela cujos valores apresentam
estatística de teste igual a zero (aqui apresentado pelo teste
implementado como \texttt{NaN}), apresentam o maior p-valor, o qual
calculado manualmente corresponde a 1. De fato, faz sentido com a
interpretação de que não há um posicionamento consistente dos rankings
dos tratamentos entre os blocos.

\begin{longtable}{ccccc}
\toprule
Tratamento & Bloco 1 & Bloco 2 & Bloco3 & Bloco 4\\
\midrule
\endfirsthead
\multicolumn{5}{@{}l}{\textit{(continued)}}\\
\toprule
Tratamento & Bloco 1 & Bloco 2 & Bloco3 & Bloco 4\\
\midrule
\endhead

\endfoot
\bottomrule
\endlastfoot
\cellcolor{gray!15}{1} & \cellcolor{gray!15}{1} & \cellcolor{gray!15}{1} & \cellcolor{gray!15}{1} & \cellcolor{gray!15}{1}\\
2 & 2 & 2 & 2 & 2\\
\cellcolor{gray!15}{3} & \cellcolor{gray!15}{3} & \cellcolor{gray!15}{3} & \cellcolor{gray!15}{3} & \cellcolor{gray!15}{3}\\
4 & 4 & 4 & 4 & 4\\*
\end{longtable}

\begin{longtable}{cccc}
\toprule
statistic & p.value & parameter & method\\
\midrule
\endfirsthead
\multicolumn{4}{@{}l}{\textit{(continued)}}\\
\toprule
statistic & p.value & parameter & method\\
\midrule
\endhead

\endfoot
\bottomrule
\endlastfoot
\cellcolor{gray!15}{12} & \cellcolor{gray!15}{0.0073832} & \cellcolor{gray!15}{3} & \cellcolor{gray!15}{Friedman rank sum test}\\*
\end{longtable}

\begin{longtable}{ccccc}
\toprule
Tratamento & Bloco 1 & Bloco 2 & Bloco3 & Bloco 4\\
\midrule
\endfirsthead
\multicolumn{5}{@{}l}{\textit{(continued)}}\\
\toprule
Tratamento & Bloco 1 & Bloco 2 & Bloco3 & Bloco 4\\
\midrule
\endhead

\endfoot
\bottomrule
\endlastfoot
\cellcolor{gray!15}{1} & \cellcolor{gray!15}{1} & \cellcolor{gray!15}{2} & \cellcolor{gray!15}{3} & \cellcolor{gray!15}{4}\\
2 & 1 & 2 & 3 & 4\\
\cellcolor{gray!15}{3} & \cellcolor{gray!15}{1} & \cellcolor{gray!15}{2} & \cellcolor{gray!15}{3} & \cellcolor{gray!15}{4}\\
4 & 1 & 2 & 3 & 4\\*
\end{longtable}

\begin{longtable}{cccc}
\toprule
statistic & p.value & parameter & method\\
\midrule
\endfirsthead
\multicolumn{4}{@{}l}{\textit{(continued)}}\\
\toprule
statistic & p.value & parameter & method\\
\midrule
\endhead

\endfoot
\bottomrule
\endlastfoot
\cellcolor{gray!15}{NaN} & \cellcolor{gray!15}{NaN} & \cellcolor{gray!15}{3} & \cellcolor{gray!15}{Friedman rank sum test}\\*
\end{longtable}

\end{document}
